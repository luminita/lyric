\documentclass[12pt]{article}
\usepackage{graphicx}
\usepackage[margin=1in]{geometry}
%\usepackage{float}
\usepackage{multirow}
\usepackage{fancyhdr}
\usepackage{listings}

\title{{\bf Music and poetry with spotify - Assignment solution}}

\author{
Luminita Moruz  \\
\hline
}

\begin{document}
\maketitle

\section{Problem formulation \& assumptions}
\label{sec:problem}
An algorithm should be implemented that takes as input a message, and
returns a list of Spotify tracks recreating this message. The
following requirements should be incorporated: the message should be
UTF-8, no case sensitivity, the list of tracks returned should be
short and the number of queries should be reduced to improve speed.

\vspace{0.15cm}

In this implementations, a few simplifications were made:
\begin{itemize}
\item {\bf Only exact matches were considered}. Any fragment of the
  message had to match precisely the full name of a track.

\item {\bf The message was split by punctuation marks}. To improve
  semantics, the message was split in sentences that are treated
  separately.

\item {\bf The algorithm gives a results only when a full solution e}
\end{itemize}
 

\section{Implemented solution}
First, the message is split at any occurence of at any occurence of
\{``.'', ``;'', ``:'', ``,''\}. Each resulted sentence is processed
separately as described below.

Let $S = w_0, w_1, ..., w_{n-1}$ be a sentence composed of $n$ words. I 


\section{Possible improvements}
\begin{itemize}
\item Improve the splitting of the poem in sentences

\item Partial solutions should be provided 

\item Inexact matches should be considered A list of the most
  commonly used abbreviations, as well as typos can be used to provide
  an answer even when no exact matches are available. 

\item The list of tracks should match the feeling of the poem
\end{itemize}
\end{document}
